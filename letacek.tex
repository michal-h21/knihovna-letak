\section{Co knihovna nabízí?}

% \bigskip
\subsection{Výpůjční protokol}

% \bigskip
\smallsection{Půjčování
dokumentů}

Většina knihovního fondu je uložena ve skladu, menší část
v~depozitářích. Ze skladu se půjčuje na počkání, z~depozitářů na
objednávku do dalšího dne nebo k~dohodnutému datu.

\smallsection{Centrální katalog UK}

Požadované tituly se vyhledávají v~\href{http://ckis.cuni.cz/}{\emph{Centrálním katalogu UK}}. Každý
dokument je označen signaturou, kterou je potřeba si zaznamenat a předat
pracovníkovi knihovny. Každý uživatel má v~katalogu vytvořený vlastní
účet, jehož pomocí může prodlužovat výpůjčky, rezervovat dokumenty nebo
sledovat stav konta.

K~práci s vlastním uživatelským kontem v~katalogu je třeba mít
aktivovaný uživatelský účet v
\href{https://ldap1.cuni.cz/}{\emph{Centrální autentizační službě}},
přístupové údaje dostanete při vystavení uživatelského průkazu UK.

\smallsection{Vracení knih}

Standardní výpůjční lhůta je 30 dní, knihy je možné prodloužit.
K~vracení knih můžete také využít Bibliobox, který se nachází v~přízemí
u vchodu do Pedagogické fakulty a je dostupný po celou dobu provozu
budovy.

\smallsection{Registrace}

K~tomu, abyste mohli využívat služeb knihovny, je nutné se
zaregistrovat. To můžete provést při první osobní návštěvě ve výpůjčním
protokolu.

K~registraci budete potřebovat:

\begin{itemize}
\item
  průkaz UK (vystavuje na počkání
  \href{http://www.cuni.cz/UK-3249.html}{\emph{Informační a poradenské
  centrum UK}}),
\item
  v případě studentů UK také platný kupón k aktuálnímu akademickému
  roku.
\end{itemize}

\newpage
\subsection{Studovna}

Pro prezenční studium dokumentů je určena studovna knihovny. Tu mohou
využívat studenti s~platným průkazem UK, ale i externí uživatelé (za
poplatek). Uloženy jsou zde odborné knihy, odborné i populárně naučné
časopisy, učebnice, skripta nebo noviny.

Ve studovně jsou také k dispozici počítače s přístupem na internet,
samoobslužné tiskárny, kopírky a skener, nabízíme také možnost kroužkové
vazby.

Můžete využít i doplňkových služeb a vypůjčit si USB nabíječku na
telefon, čtečky elektronických knih, deskové hry nebo flash disk (vše
v~rámci studovny).

% \subsection{Další služby a nástroje}

\smallsection{Služby pro studenty se speciálními
potřebami}

Ve studovně se nachází vyčleněný počítač s~asistenčním softwarem
a~možností tisku a kopírování zdarma. Pro studenty se zrakovým
postižením nabízíme přístup ke knihám v~elektronické podobě a kamerovou
zvětšovací lupu. Pro studenty na vozíčku jsou ve studovně poskytovány výpůjční
služby.

\smallsection{Elektronické informační
zdroje}

Univerzita Karlova umožňuje pro studenty a zaměstnance přístup
k~elektronickým informačním zdrojům. V~nabídce jsou elektronické
časopisy, články nebo knihy různých oborů. Jsou dostupné přímo
z~fakultní sítě nebo pomocí vzdáleného přístupu (s využitím
přihlašovacích údajů do CAS).

Více informací naleznete např. na Portálu elektronických zdrojů
{\href{http://pez.cuni.cz}{Univerzity Karlovy}} nebo vyhledávací
službě UKAŽ.

\smallsection{Citace PRO }

Pro usnadnění tvorby a správy citací literatury knihovna nabízí citační
manažer Citace PRO. Studenti a zaměstnanci fakulty ho mohou využívat
zdarma. Pro další informace sledujte webové stránky knihovny nebo
kontaktujte Martinu Růžičkovou
\mail{martina.ruzickova@pedf.cuni.cz}.

\section{Kde najdete knihovnu?}

Knihovna se nachází v hlavní~budově PedF UK (Magdaleny Rettigové 4,
Praha 1) v~přízemí.

% \textbf{Jak se k~nám dostanete?}

% \begin{itemize}[leftmargin=0pt, topsep=0pt]
% \item
%   metrem B, stanice Národní třída -- výtahem přímo do ulice Magdalény
%   Rettigové
% \item
%   tramvají č. 3, 6, 9, 14, 24 zastávka Lazarská
% \end{itemize}

\smallsection{Knihovna je rozdělena na dvě části:}

\begin{itemize}[leftmargin=0pt, topsep=0pt]
\item Výpůjční protokol se nachází napravo od hlavního vchodu
  do budovy, přes dvorek, vedle zadního vchodu do Auly.
\item Studovna  je na levé straně budovy.
\end{itemize}

  \rotatebox{90}{% Graphic for TeX using PGF
% Title: /home/mint/knihovna/planekknihovny.dia
% Creator: Dia v0.97.3
% CreationDate: Tue Oct 10 14:49:24 2017
% For: mint
% \usepackage{tikz}
% The following commands are not supported in PSTricks at present
% We define them conditionally, so when they are implemented,
% this pgf file will use them.
\ifx\du\undefined
  \newlength{\du}
\fi
\setlength{\du}{11\unitlength}
\begin{tikzpicture}
\pgftransformxscale{1.000000}
\pgftransformyscale{-1.000000}
\definecolor{dialinecolor}{rgb}{0.000000, 0.000000, 0.000000}
\pgfsetstrokecolor{dialinecolor}
\definecolor{dialinecolor}{rgb}{1.000000, 1.000000, 1.000000}
\pgfsetfillcolor{dialinecolor}
\definecolor{dialinecolor}{rgb}{1.000000, 1.000000, 1.000000}
\pgfsetfillcolor{dialinecolor}
\fill (0.300000\du,4.300000\du)--(0.300000\du,25.750000\du)--(31.700000\du,25.750000\du)--(31.700000\du,4.300000\du)--cycle;
\pgfsetlinewidth{0.100000\du}
\pgfsetdash{}{0pt}
\pgfsetdash{}{0pt}
\pgfsetmiterjoin
\definecolor{dialinecolor}{rgb}{0.000000, 0.000000, 0.000000}
\pgfsetstrokecolor{dialinecolor}
\draw (0.300000\du,4.300000\du)--(0.300000\du,25.750000\du)--(31.700000\du,25.750000\du)--(31.700000\du,4.300000\du)--cycle;
% setfont left to latex
\definecolor{dialinecolor}{rgb}{0.000000, 0.000000, 0.000000}
\pgfsetstrokecolor{dialinecolor}
\node at (16.000000\du,15.310000\du){};
\definecolor{dialinecolor}{rgb}{1.000000, 1.000000, 1.000000}
\pgfsetfillcolor{dialinecolor}
\fill (6.100000\du,8.600000\du)--(6.100000\du,20.700000\du)--(15.300000\du,20.700000\du)--(15.300000\du,8.600000\du)--cycle;
\pgfsetlinewidth{0.100000\du}
\pgfsetdash{}{0pt}
\pgfsetdash{}{0pt}
\pgfsetmiterjoin
\definecolor{dialinecolor}{rgb}{0.000000, 0.000000, 0.000000}
\pgfsetstrokecolor{dialinecolor}
\draw (6.100000\du,8.600000\du)--(6.100000\du,20.700000\du)--(15.300000\du,20.700000\du)--(15.300000\du,8.600000\du)--cycle;
% setfont left to latex
\definecolor{dialinecolor}{rgb}{0.000000, 0.000000, 0.000000}
\pgfsetstrokecolor{dialinecolor}
\node at (10.700000\du,14.935000\du){Dvorek};
\definecolor{dialinecolor}{rgb}{1.000000, 1.000000, 1.000000}
\pgfsetfillcolor{dialinecolor}
\fill (15.350000\du,4.350000\du)--(15.350000\du,20.700000\du)--(24.150000\du,20.700000\du)--(24.150000\du,4.350000\du)--cycle;
\pgfsetlinewidth{0.100000\du}
\pgfsetdash{}{0pt}
\pgfsetdash{}{0pt}
\pgfsetmiterjoin
\definecolor{dialinecolor}{rgb}{0.000000, 0.000000, 0.000000}
\pgfsetstrokecolor{dialinecolor}
\draw (15.350000\du,4.350000\du)--(15.350000\du,20.700000\du)--(24.150000\du,20.700000\du)--(24.150000\du,4.350000\du)--cycle;
% setfont left to latex
\definecolor{dialinecolor}{rgb}{0.000000, 0.000000, 0.000000}
\pgfsetstrokecolor{dialinecolor}
\node at (19.750000\du,12.810000\du){Aula};
\pgfsetlinewidth{0.100000\du}
\pgfsetdash{}{0pt}
\pgfsetdash{}{0pt}
\pgfsetbuttcap
\pgfsetmiterjoin
\pgfsetlinewidth{0.100000\du}
\pgfsetbuttcap
\pgfsetmiterjoin
\pgfsetdash{}{0pt}
\definecolor{dialinecolor}{rgb}{1.000000, 1.000000, 1.000000}
\pgfsetfillcolor{dialinecolor}
\pgfpathellipse{\pgfpoint{11.300000\du}{17.650000\du}}{\pgfpoint{0.700000\du}{0\du}}{\pgfpoint{0\du}{0.700000\du}}
\pgfusepath{fill}
\definecolor{dialinecolor}{rgb}{0.000000, 0.000000, 0.000000}
\pgfsetstrokecolor{dialinecolor}
\pgfpathellipse{\pgfpoint{11.300000\du}{17.650000\du}}{\pgfpoint{0.700000\du}{0\du}}{\pgfpoint{0\du}{0.700000\du}}
\pgfusepath{stroke}
\pgfsetbuttcap
\pgfsetmiterjoin
\pgfsetdash{}{0pt}
\definecolor{dialinecolor}{rgb}{0.000000, 0.000000, 0.000000}
\pgfsetstrokecolor{dialinecolor}
\draw (11.300000\du,16.950000\du)--(11.300000\du,18.350000\du);
\pgfsetbuttcap
\pgfsetmiterjoin
\pgfsetdash{}{0pt}
\definecolor{dialinecolor}{rgb}{0.000000, 0.000000, 0.000000}
\pgfsetstrokecolor{dialinecolor}
\draw (10.600000\du,17.650000\du)--(12.000000\du,17.650000\du);
\pgfsetlinewidth{0.100000\du}
\pgfsetdash{}{0pt}
\pgfsetdash{}{0pt}
\pgfsetbuttcap
\pgfsetmiterjoin
\pgfsetlinewidth{0.100000\du}
\pgfsetbuttcap
\pgfsetmiterjoin
\pgfsetdash{}{0pt}
\definecolor{dialinecolor}{rgb}{1.000000, 1.000000, 1.000000}
\pgfsetfillcolor{dialinecolor}
\pgfpathellipse{\pgfpoint{11.200000\du}{11.725000\du}}{\pgfpoint{0.650000\du}{0\du}}{\pgfpoint{0\du}{0.625000\du}}
\pgfusepath{fill}
\definecolor{dialinecolor}{rgb}{0.000000, 0.000000, 0.000000}
\pgfsetstrokecolor{dialinecolor}
\pgfpathellipse{\pgfpoint{11.200000\du}{11.725000\du}}{\pgfpoint{0.650000\du}{0\du}}{\pgfpoint{0\du}{0.625000\du}}
\pgfusepath{stroke}
\pgfsetbuttcap
\pgfsetmiterjoin
\pgfsetdash{}{0pt}
\definecolor{dialinecolor}{rgb}{0.000000, 0.000000, 0.000000}
\pgfsetstrokecolor{dialinecolor}
\draw (11.200000\du,11.100000\du)--(11.200000\du,12.350000\du);
\pgfsetbuttcap
\pgfsetmiterjoin
\pgfsetdash{}{0pt}
\definecolor{dialinecolor}{rgb}{0.000000, 0.000000, 0.000000}
\pgfsetstrokecolor{dialinecolor}
\draw (10.550000\du,11.725000\du)--(11.850000\du,11.725000\du);
\definecolor{dialinecolor}{rgb}{1.000000, 1.000000, 1.000000}
\pgfsetfillcolor{dialinecolor}
\fill (24.153750\du,8.500000\du)--(24.153750\du,20.700000\du)--(31.700000\du,20.700000\du)--(31.700000\du,8.500000\du)--cycle;
\pgfsetlinewidth{0.100000\du}
\pgfsetdash{}{0pt}
\pgfsetdash{}{0pt}
\pgfsetmiterjoin
\definecolor{dialinecolor}{rgb}{0.000000, 0.000000, 0.000000}
\pgfsetstrokecolor{dialinecolor}
\draw (24.153750\du,8.500000\du)--(24.153750\du,20.700000\du)--(31.700000\du,20.700000\du)--(31.700000\du,8.500000\du)--cycle;
% setfont left to latex
\definecolor{dialinecolor}{rgb}{0.000000, 0.000000, 0.000000}
\pgfsetstrokecolor{dialinecolor}
\node at (27.926875\du,14.885000\du){Dvorek};
\definecolor{dialinecolor}{rgb}{1.000000, 1.000000, 1.000000}
\pgfsetfillcolor{dialinecolor}
\fill (1.707500\du,8.600000\du)--(1.707500\du,20.650000\du)--(6.050000\du,20.650000\du)--(6.050000\du,8.600000\du)--cycle;
\pgfsetlinewidth{0.100000\du}
\pgfsetdash{}{0pt}
\pgfsetdash{}{0pt}
\pgfsetmiterjoin
\definecolor{dialinecolor}{rgb}{0.000000, 0.000000, 0.000000}
\pgfsetstrokecolor{dialinecolor}
\draw (1.707500\du,8.600000\du)--(1.707500\du,20.650000\du)--(6.050000\du,20.650000\du)--(6.050000\du,8.600000\du)--cycle;
% setfont left to latex
\definecolor{dialinecolor}{rgb}{0.000000, 0.000000, 0.000000}
\pgfsetstrokecolor{dialinecolor}
\node at (3.878750\du,14.910000\du){Studovna};
\pgfsetlinewidth{0.100000\du}
\pgfsetdash{}{0pt}
\pgfsetdash{}{0pt}
\pgfsetmiterjoin
\definecolor{dialinecolor}{rgb}{1.000000, 1.000000, 1.000000}
\pgfsetfillcolor{dialinecolor}
\fill (25.350000\du,20.200000\du)--(25.350000\du,20.350000\du)--(26.700000\du,20.350000\du)--(26.700000\du,20.200000\du)--cycle;
\definecolor{dialinecolor}{rgb}{0.000000, 0.000000, 0.000000}
\pgfsetstrokecolor{dialinecolor}
\draw (25.350000\du,20.200000\du)--(25.350000\du,20.750000\du)--(26.700000\du,20.750000\du)--(26.700000\du,20.200000\du)--cycle;
\pgfsetlinewidth{0.100000\du}
\pgfsetdash{}{0pt}
\pgfsetdash{}{0pt}
\pgfsetmiterjoin
\definecolor{dialinecolor}{rgb}{1.000000, 1.000000, 1.000000}
\pgfsetfillcolor{dialinecolor}
\fill (29.300000\du,8.550000\du)--(29.300000\du,9.200000\du)--(30.450000\du,9.200000\du)--(30.450000\du,8.550000\du)--cycle;
\definecolor{dialinecolor}{rgb}{0.000000, 0.000000, 0.000000}
\pgfsetstrokecolor{dialinecolor}
\draw (29.300000\du,8.550000\du)--(29.300000\du,9.200000\du)--(30.450000\du,9.200000\du)--(30.450000\du,8.550000\du)--cycle;
\pgfsetlinewidth{0.100000\du}
\pgfsetdash{}{0pt}
\pgfsetdash{}{0pt}
\pgfsetmiterjoin
\definecolor{dialinecolor}{rgb}{1.000000, 1.000000, 1.000000}
\pgfsetfillcolor{dialinecolor}
\fill (25.150000\du,4.200000\du)--(25.150000\du,4.650000\du)--(26.450000\du,4.650000\du)--(26.450000\du,4.200000\du)--cycle;
\definecolor{dialinecolor}{rgb}{0.000000, 0.000000, 0.000000}
\pgfsetstrokecolor{dialinecolor}
\draw (25.150000\du,4.300000\du)--(25.150000\du,4.650000\du)--(26.450000\du,4.650000\du)--(26.450000\du,4.300000\du)--cycle;
\pgfsetlinewidth{0.100000\du}
\pgfsetdash{}{0pt}
\pgfsetdash{}{0pt}
\pgfsetmiterjoin
\definecolor{dialinecolor}{rgb}{1.000000, 1.000000, 1.000000}
\pgfsetfillcolor{dialinecolor}
\fill (4.300000\du,8.650000\du)--(4.300000\du,9.350000\du)--(5.350000\du,9.350000\du)--(5.350000\du,8.650000\du)--cycle;
\definecolor{dialinecolor}{rgb}{0.000000, 0.000000, 0.000000}
\pgfsetstrokecolor{dialinecolor}
\draw (4.300000\du,8.650000\du)--(4.300000\du,9.350000\du)--(5.350000\du,9.350000\du)--(5.350000\du,8.650000\du)--cycle;
\pgfsetlinewidth{0.100000\du}
\pgfsetdash{}{0pt}
\pgfsetdash{}{0pt}
\pgfsetmiterjoin
\definecolor{dialinecolor}{rgb}{1.000000, 1.000000, 1.000000}
\pgfsetfillcolor{dialinecolor}
\fill (16.500000\du,25.200000\du)--(16.500000\du,25.750000\du)--(18.400000\du,25.750000\du)--(18.400000\du,25.200000\du)--cycle;
\definecolor{dialinecolor}{rgb}{0.000000, 0.000000, 0.000000}
\pgfsetstrokecolor{dialinecolor}
\draw (16.500000\du,25.200000\du)--(16.500000\du,25.750000\du)--(18.400000\du,25.750000\du)--(18.400000\du,25.200000\du)--cycle;
% setfont left to latex
\definecolor{dialinecolor}{rgb}{0.000000, 0.000000, 0.000000}
\pgfsetstrokecolor{dialinecolor}
\node[anchor=west] at (16.000000\du,15.025000\du){};
% setfont left to latex
\definecolor{dialinecolor}{rgb}{0.000000, 0.000000, 0.000000}
\pgfsetstrokecolor{dialinecolor}
\node[anchor=west] at (23.850000\du,5.350000\du){Výpůjční protokol};
% setfont left to latex
\definecolor{dialinecolor}{rgb}{0.000000, 0.000000, 0.000000}
\pgfsetstrokecolor{dialinecolor}
\node[anchor=west] at (13.500000\du,24.550000\du){Hlavní vchod do budovy};
\pgfsetlinewidth{0.100000\du}
\pgfsetdash{}{0pt}
\pgfsetdash{}{0pt}
\pgfsetmiterjoin
\definecolor{dialinecolor}{rgb}{1.000000, 1.000000, 1.000000}
\pgfsetfillcolor{dialinecolor}
\fill (7.750000\du,25.300000\du)--(7.750000\du,25.650000\du)--(9.550000\du,25.650000\du)--(9.550000\du,25.300000\du)--cycle;
\definecolor{dialinecolor}{rgb}{0.000000, 0.000000, 0.000000}
\pgfsetstrokecolor{dialinecolor}
\draw (7.750000\du,25.300000\du)--(7.750000\du,25.750000\du)--(9.550000\du,25.750000\du)--(9.550000\du,25.300000\du)--cycle;
% setfont left to latex
\definecolor{dialinecolor}{rgb}{0.000000, 0.000000, 0.000000}
\pgfsetstrokecolor{dialinecolor}
\node[anchor=west] at (7.400000\du,24.650000\du){Výtah};
\end{tikzpicture}
}

\section{Otevírací doba}

\noindent\smallsection{Výpůjční
protokol}

\ifdefined\HCode
\begin{tabular}{lllll}
  Po      & ~ & 8.00 &--& 16.00\\
  Út--pá  & ~ & 8.00 &--& 17.00
\end{tabular}

\else
Po 8.00 -- 16.00

Út 8.00 -- 17.00

St 8.00 -- 17.00

Čt 8.00 -- 17.00

Pá 8.00 -- 17.00
\fi

\noindent\smallsection{Studovna}

\ifdefined\HCode
\begin{tabular}{lllll}
  Po--pá & ~ & 8.00 &--& 18.00\\
  Pá     & ~ & 8.00 &--& 16.00
\end{tabular}
\else
Po 8.00 -- 18.00

Út 8.00 -- 18.00

St 8.00 -- 18.00

Čt 8.00 -- 18.00

Pá 8.00 -- 16.00
\fi

\newpage
\section{Kontakty}

\begin{tabular}{@{}ll@{}}
  E-mail:& \url{knihovna@pedf.cuni.cz}\\

  Výpůjční protokol:& 221~900~148\\

  Studovna:& 221~900~178\\
  Studenti se SP:& 221~900~122\\
  Facebook & \url{@knihovnapedfpraha}\\
  Instagram & \url{@knihovnapedfpraha}
\end{tabular}

\subsection{Odkazy}

\begin{tabular}{@{}ll@{}}
  Knihovna PedF UK& \url{knihovna.pedf.cuni.cz} \\

  Centrální katalog UK& \url{ckis.cuni.cz} \\

  Centrální autentizační služba& \url{cas.cuni.cz} \\

  Portál elektronických zdrojů& \url{pez.cuni.cz} \\

  Vyhledávač UKAŽ & \url{ukaz.cuni.cz} \\

  Citace PRO & \url{citace.com/citace-pro} \\

  Ústřední knihovna UK & \url{knihovna.cuni.cz} \\

  IPSC UK & \url{ipc.cuni.cz} 
\end{tabular}
