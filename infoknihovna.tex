\documentclass[12]{leaflet}

\usepackage{fontspec}
\setmainfont{Fira Sans}[Ligatures=TeX]
\usepackage{uklogo}
% \usepackage[T1]{fontenc}
% \usepackage[utf8]{inputenc}
\usepackage[czech]{babel}
\usepackage{url}
\usepackage{tabularx}
\usepackage{tikz}
\usepackage{graphicx}


\newenvironment{page}{\easylist}{\endeasylist\clearpage}
\usepackage[at]{easylist}
\ListProperties(Hide=10)
\newcommand\potrebujes{\textbf{Potřebujete:}\par\bigskip}
\newcommand\mytitle[1]{\bgroup\large\bfseries #1\egroup}
\renewcommand\subsection[1]{\par\vskip3pt\bgroup\large #1\egroup\par}
\newcommand\vykrik[1]{\medskip\par\bgroup\Large\MakeUppercase{#1}\egroup}
\newenvironment{info}{}{}
\begin{document}


\raggedright
\begin{page}
  \uklogo[64pt]
  \vfill
  \url{knihovna.pedf.cuni.cz}
  \rotatebox{90}{% Graphic for TeX using PGF
% Title: /home/mint/knihovna/planekknihovny.dia
% Creator: Dia v0.97.3
% CreationDate: Tue Oct 10 14:49:24 2017
% For: mint
% \usepackage{tikz}
% The following commands are not supported in PSTricks at present
% We define them conditionally, so when they are implemented,
% this pgf file will use them.
\ifx\du\undefined
  \newlength{\du}
\fi
\setlength{\du}{11\unitlength}
\begin{tikzpicture}
\pgftransformxscale{1.000000}
\pgftransformyscale{-1.000000}
\definecolor{dialinecolor}{rgb}{0.000000, 0.000000, 0.000000}
\pgfsetstrokecolor{dialinecolor}
\definecolor{dialinecolor}{rgb}{1.000000, 1.000000, 1.000000}
\pgfsetfillcolor{dialinecolor}
\definecolor{dialinecolor}{rgb}{1.000000, 1.000000, 1.000000}
\pgfsetfillcolor{dialinecolor}
\fill (0.300000\du,4.300000\du)--(0.300000\du,25.750000\du)--(31.700000\du,25.750000\du)--(31.700000\du,4.300000\du)--cycle;
\pgfsetlinewidth{0.100000\du}
\pgfsetdash{}{0pt}
\pgfsetdash{}{0pt}
\pgfsetmiterjoin
\definecolor{dialinecolor}{rgb}{0.000000, 0.000000, 0.000000}
\pgfsetstrokecolor{dialinecolor}
\draw (0.300000\du,4.300000\du)--(0.300000\du,25.750000\du)--(31.700000\du,25.750000\du)--(31.700000\du,4.300000\du)--cycle;
% setfont left to latex
\definecolor{dialinecolor}{rgb}{0.000000, 0.000000, 0.000000}
\pgfsetstrokecolor{dialinecolor}
\node at (16.000000\du,15.310000\du){};
\definecolor{dialinecolor}{rgb}{1.000000, 1.000000, 1.000000}
\pgfsetfillcolor{dialinecolor}
\fill (6.100000\du,8.600000\du)--(6.100000\du,20.700000\du)--(15.300000\du,20.700000\du)--(15.300000\du,8.600000\du)--cycle;
\pgfsetlinewidth{0.100000\du}
\pgfsetdash{}{0pt}
\pgfsetdash{}{0pt}
\pgfsetmiterjoin
\definecolor{dialinecolor}{rgb}{0.000000, 0.000000, 0.000000}
\pgfsetstrokecolor{dialinecolor}
\draw (6.100000\du,8.600000\du)--(6.100000\du,20.700000\du)--(15.300000\du,20.700000\du)--(15.300000\du,8.600000\du)--cycle;
% setfont left to latex
\definecolor{dialinecolor}{rgb}{0.000000, 0.000000, 0.000000}
\pgfsetstrokecolor{dialinecolor}
\node at (10.700000\du,14.935000\du){Dvorek};
\definecolor{dialinecolor}{rgb}{1.000000, 1.000000, 1.000000}
\pgfsetfillcolor{dialinecolor}
\fill (15.350000\du,4.350000\du)--(15.350000\du,20.700000\du)--(24.150000\du,20.700000\du)--(24.150000\du,4.350000\du)--cycle;
\pgfsetlinewidth{0.100000\du}
\pgfsetdash{}{0pt}
\pgfsetdash{}{0pt}
\pgfsetmiterjoin
\definecolor{dialinecolor}{rgb}{0.000000, 0.000000, 0.000000}
\pgfsetstrokecolor{dialinecolor}
\draw (15.350000\du,4.350000\du)--(15.350000\du,20.700000\du)--(24.150000\du,20.700000\du)--(24.150000\du,4.350000\du)--cycle;
% setfont left to latex
\definecolor{dialinecolor}{rgb}{0.000000, 0.000000, 0.000000}
\pgfsetstrokecolor{dialinecolor}
\node at (19.750000\du,12.810000\du){Aula};
\pgfsetlinewidth{0.100000\du}
\pgfsetdash{}{0pt}
\pgfsetdash{}{0pt}
\pgfsetbuttcap
\pgfsetmiterjoin
\pgfsetlinewidth{0.100000\du}
\pgfsetbuttcap
\pgfsetmiterjoin
\pgfsetdash{}{0pt}
\definecolor{dialinecolor}{rgb}{1.000000, 1.000000, 1.000000}
\pgfsetfillcolor{dialinecolor}
\pgfpathellipse{\pgfpoint{11.300000\du}{17.650000\du}}{\pgfpoint{0.700000\du}{0\du}}{\pgfpoint{0\du}{0.700000\du}}
\pgfusepath{fill}
\definecolor{dialinecolor}{rgb}{0.000000, 0.000000, 0.000000}
\pgfsetstrokecolor{dialinecolor}
\pgfpathellipse{\pgfpoint{11.300000\du}{17.650000\du}}{\pgfpoint{0.700000\du}{0\du}}{\pgfpoint{0\du}{0.700000\du}}
\pgfusepath{stroke}
\pgfsetbuttcap
\pgfsetmiterjoin
\pgfsetdash{}{0pt}
\definecolor{dialinecolor}{rgb}{0.000000, 0.000000, 0.000000}
\pgfsetstrokecolor{dialinecolor}
\draw (11.300000\du,16.950000\du)--(11.300000\du,18.350000\du);
\pgfsetbuttcap
\pgfsetmiterjoin
\pgfsetdash{}{0pt}
\definecolor{dialinecolor}{rgb}{0.000000, 0.000000, 0.000000}
\pgfsetstrokecolor{dialinecolor}
\draw (10.600000\du,17.650000\du)--(12.000000\du,17.650000\du);
\pgfsetlinewidth{0.100000\du}
\pgfsetdash{}{0pt}
\pgfsetdash{}{0pt}
\pgfsetbuttcap
\pgfsetmiterjoin
\pgfsetlinewidth{0.100000\du}
\pgfsetbuttcap
\pgfsetmiterjoin
\pgfsetdash{}{0pt}
\definecolor{dialinecolor}{rgb}{1.000000, 1.000000, 1.000000}
\pgfsetfillcolor{dialinecolor}
\pgfpathellipse{\pgfpoint{11.200000\du}{11.725000\du}}{\pgfpoint{0.650000\du}{0\du}}{\pgfpoint{0\du}{0.625000\du}}
\pgfusepath{fill}
\definecolor{dialinecolor}{rgb}{0.000000, 0.000000, 0.000000}
\pgfsetstrokecolor{dialinecolor}
\pgfpathellipse{\pgfpoint{11.200000\du}{11.725000\du}}{\pgfpoint{0.650000\du}{0\du}}{\pgfpoint{0\du}{0.625000\du}}
\pgfusepath{stroke}
\pgfsetbuttcap
\pgfsetmiterjoin
\pgfsetdash{}{0pt}
\definecolor{dialinecolor}{rgb}{0.000000, 0.000000, 0.000000}
\pgfsetstrokecolor{dialinecolor}
\draw (11.200000\du,11.100000\du)--(11.200000\du,12.350000\du);
\pgfsetbuttcap
\pgfsetmiterjoin
\pgfsetdash{}{0pt}
\definecolor{dialinecolor}{rgb}{0.000000, 0.000000, 0.000000}
\pgfsetstrokecolor{dialinecolor}
\draw (10.550000\du,11.725000\du)--(11.850000\du,11.725000\du);
\definecolor{dialinecolor}{rgb}{1.000000, 1.000000, 1.000000}
\pgfsetfillcolor{dialinecolor}
\fill (24.153750\du,8.500000\du)--(24.153750\du,20.700000\du)--(31.700000\du,20.700000\du)--(31.700000\du,8.500000\du)--cycle;
\pgfsetlinewidth{0.100000\du}
\pgfsetdash{}{0pt}
\pgfsetdash{}{0pt}
\pgfsetmiterjoin
\definecolor{dialinecolor}{rgb}{0.000000, 0.000000, 0.000000}
\pgfsetstrokecolor{dialinecolor}
\draw (24.153750\du,8.500000\du)--(24.153750\du,20.700000\du)--(31.700000\du,20.700000\du)--(31.700000\du,8.500000\du)--cycle;
% setfont left to latex
\definecolor{dialinecolor}{rgb}{0.000000, 0.000000, 0.000000}
\pgfsetstrokecolor{dialinecolor}
\node at (27.926875\du,14.885000\du){Dvorek};
\definecolor{dialinecolor}{rgb}{1.000000, 1.000000, 1.000000}
\pgfsetfillcolor{dialinecolor}
\fill (1.707500\du,8.600000\du)--(1.707500\du,20.650000\du)--(6.050000\du,20.650000\du)--(6.050000\du,8.600000\du)--cycle;
\pgfsetlinewidth{0.100000\du}
\pgfsetdash{}{0pt}
\pgfsetdash{}{0pt}
\pgfsetmiterjoin
\definecolor{dialinecolor}{rgb}{0.000000, 0.000000, 0.000000}
\pgfsetstrokecolor{dialinecolor}
\draw (1.707500\du,8.600000\du)--(1.707500\du,20.650000\du)--(6.050000\du,20.650000\du)--(6.050000\du,8.600000\du)--cycle;
% setfont left to latex
\definecolor{dialinecolor}{rgb}{0.000000, 0.000000, 0.000000}
\pgfsetstrokecolor{dialinecolor}
\node at (3.878750\du,14.910000\du){Studovna};
\pgfsetlinewidth{0.100000\du}
\pgfsetdash{}{0pt}
\pgfsetdash{}{0pt}
\pgfsetmiterjoin
\definecolor{dialinecolor}{rgb}{1.000000, 1.000000, 1.000000}
\pgfsetfillcolor{dialinecolor}
\fill (25.350000\du,20.200000\du)--(25.350000\du,20.350000\du)--(26.700000\du,20.350000\du)--(26.700000\du,20.200000\du)--cycle;
\definecolor{dialinecolor}{rgb}{0.000000, 0.000000, 0.000000}
\pgfsetstrokecolor{dialinecolor}
\draw (25.350000\du,20.200000\du)--(25.350000\du,20.750000\du)--(26.700000\du,20.750000\du)--(26.700000\du,20.200000\du)--cycle;
\pgfsetlinewidth{0.100000\du}
\pgfsetdash{}{0pt}
\pgfsetdash{}{0pt}
\pgfsetmiterjoin
\definecolor{dialinecolor}{rgb}{1.000000, 1.000000, 1.000000}
\pgfsetfillcolor{dialinecolor}
\fill (29.300000\du,8.550000\du)--(29.300000\du,9.200000\du)--(30.450000\du,9.200000\du)--(30.450000\du,8.550000\du)--cycle;
\definecolor{dialinecolor}{rgb}{0.000000, 0.000000, 0.000000}
\pgfsetstrokecolor{dialinecolor}
\draw (29.300000\du,8.550000\du)--(29.300000\du,9.200000\du)--(30.450000\du,9.200000\du)--(30.450000\du,8.550000\du)--cycle;
\pgfsetlinewidth{0.100000\du}
\pgfsetdash{}{0pt}
\pgfsetdash{}{0pt}
\pgfsetmiterjoin
\definecolor{dialinecolor}{rgb}{1.000000, 1.000000, 1.000000}
\pgfsetfillcolor{dialinecolor}
\fill (25.150000\du,4.200000\du)--(25.150000\du,4.650000\du)--(26.450000\du,4.650000\du)--(26.450000\du,4.200000\du)--cycle;
\definecolor{dialinecolor}{rgb}{0.000000, 0.000000, 0.000000}
\pgfsetstrokecolor{dialinecolor}
\draw (25.150000\du,4.300000\du)--(25.150000\du,4.650000\du)--(26.450000\du,4.650000\du)--(26.450000\du,4.300000\du)--cycle;
\pgfsetlinewidth{0.100000\du}
\pgfsetdash{}{0pt}
\pgfsetdash{}{0pt}
\pgfsetmiterjoin
\definecolor{dialinecolor}{rgb}{1.000000, 1.000000, 1.000000}
\pgfsetfillcolor{dialinecolor}
\fill (4.300000\du,8.650000\du)--(4.300000\du,9.350000\du)--(5.350000\du,9.350000\du)--(5.350000\du,8.650000\du)--cycle;
\definecolor{dialinecolor}{rgb}{0.000000, 0.000000, 0.000000}
\pgfsetstrokecolor{dialinecolor}
\draw (4.300000\du,8.650000\du)--(4.300000\du,9.350000\du)--(5.350000\du,9.350000\du)--(5.350000\du,8.650000\du)--cycle;
\pgfsetlinewidth{0.100000\du}
\pgfsetdash{}{0pt}
\pgfsetdash{}{0pt}
\pgfsetmiterjoin
\definecolor{dialinecolor}{rgb}{1.000000, 1.000000, 1.000000}
\pgfsetfillcolor{dialinecolor}
\fill (16.500000\du,25.200000\du)--(16.500000\du,25.750000\du)--(18.400000\du,25.750000\du)--(18.400000\du,25.200000\du)--cycle;
\definecolor{dialinecolor}{rgb}{0.000000, 0.000000, 0.000000}
\pgfsetstrokecolor{dialinecolor}
\draw (16.500000\du,25.200000\du)--(16.500000\du,25.750000\du)--(18.400000\du,25.750000\du)--(18.400000\du,25.200000\du)--cycle;
% setfont left to latex
\definecolor{dialinecolor}{rgb}{0.000000, 0.000000, 0.000000}
\pgfsetstrokecolor{dialinecolor}
\node[anchor=west] at (16.000000\du,15.025000\du){};
% setfont left to latex
\definecolor{dialinecolor}{rgb}{0.000000, 0.000000, 0.000000}
\pgfsetstrokecolor{dialinecolor}
\node[anchor=west] at (23.850000\du,5.350000\du){Výpůjční protokol};
% setfont left to latex
\definecolor{dialinecolor}{rgb}{0.000000, 0.000000, 0.000000}
\pgfsetstrokecolor{dialinecolor}
\node[anchor=west] at (13.500000\du,24.550000\du){Hlavní vchod do budovy};
\pgfsetlinewidth{0.100000\du}
\pgfsetdash{}{0pt}
\pgfsetdash{}{0pt}
\pgfsetmiterjoin
\definecolor{dialinecolor}{rgb}{1.000000, 1.000000, 1.000000}
\pgfsetfillcolor{dialinecolor}
\fill (7.750000\du,25.300000\du)--(7.750000\du,25.650000\du)--(9.550000\du,25.650000\du)--(9.550000\du,25.300000\du)--cycle;
\definecolor{dialinecolor}{rgb}{0.000000, 0.000000, 0.000000}
\pgfsetstrokecolor{dialinecolor}
\draw (7.750000\du,25.300000\du)--(7.750000\du,25.750000\du)--(9.550000\du,25.750000\du)--(9.550000\du,25.300000\du)--cycle;
% setfont left to latex
\definecolor{dialinecolor}{rgb}{0.000000, 0.000000, 0.000000}
\pgfsetstrokecolor{dialinecolor}
\node[anchor=west] at (7.400000\du,24.650000\du){Výtah};
\end{tikzpicture}
}
\end{page}
\begin{page}
 
\potrebujes
% \begin{easylist}
@ literaturu ke studiu
@ článek z časopisu
@ tisknout 
@ kopírovat
@ skenovat
@ poradit s informačními zdroji
@ poradit s citováním ?
% \end{easylist}


\vykrik{přijďte do knihovny!}

@ \potrebujes půjčit publikace domů ?

\vykrik{přijďte do Výpůjčního protokolu}

\end{page}
\begin{page}
\potrebujes 
@ knihu, která je dostupná jen ve studovně,
@ prolistovat časopisy
@ tisknout, kopírovat, skenovat
@ poradit s informačními zdroji 
\vykrik{přijď do Studovny}


\potrebujes 

@ specifické služby

\vykrik{Využijte služeb studovny}

\begin{info}
ve studovně najdeš dva počítače vybavené speciálním softwarem pro zrakově
postižené, kamerovou čtecí lupu, multifunkční
zařízení: skener, tiskárnu a kopírku. 

Seznam digitalizované literatury pro studenty se zrakovým postižením najdeš na
našem webu v sekci Služby.
\end{info}

\potrebujes 
@ najít publikace ke studiu, jejich signatury
@ zarezervovat knihu
@ prodloužit výpůjčky

\vykrik{k tomu slouží Centrální katalog UK}

          \url{ckis.cuni.cz}

\potrebujes 
@ vyhledat literaturu pro psaní práce
@ článek ze zahraničního časopisu?
\vykrik{využijte elektronické informační zdroje} 

přístup z webu knihovny na fakultě přímo, z domova po přihlášení heslem CAS

\end{page}

\begin{page}
K využívání služeb knihoven potřebuješ  Průkaz studenta UK, příp. průkaz externího uživatele služeb, …

Seznam výdejních center je na webu UK…
Při první návštěvě v Ústřední knihovně PedF  
průkaz + vyplnit Dohodu o výpůjční službě automatizovaným systémem
Průkaz platí do všech knihoven na UK, ale ve všech je třeba se registrovat a chovat podle místních knihovních řádů.

Další informace 
web ÚK PedF UK, Informační odpoledne pro studenty 1. ročníků s praktickou instruktáží, nástěnky, pracovníci knihovny, ????



Těšíme se na Tebe
zaměstnanci Ústřední knihovny PedF UK
\end{page}  

\begin{page}
\section{Kontakty}

\subsection{Výpůjční protokol}
					Magdalény Rettigové 4, přízemí,\\
vpravo přes kuřácký dvorek, dveře vedle auly\\

\smallskip

\begin{tabular}{@{}ll}
  Po & 8.00--16.00\\
  Ut--Pá & 8.00--17.00\\
\end{tabular}

\subsection{Studovna}
				Magdalény Rettigové 4, v přízemí vlevo a ještě 2x vlevo, 
pak dveře po pravé straně za regály s časopisy

\smallskip

\begin{tabular}{@{}ll}
  Po--Čt & 8.00--18.00\\
  Pá & 8.00--16.00\\
\end{tabular}

\subsection{Studovna anglického jazyka a literatury}
				Celetná 13, suterén

        \smallskip

\begin{flushleft}
\noindent\begin{tabular}{@{}ll}
Po &    13.00--16.00\\
Út & 9.00--12.00  \\
St & 9.00--12.00 \\
Čt &    --\\
Pá &    13.00--16.00 \\
\end{tabular}
      \end{flushleft}

\section{Odkazy}

\subsection{WWW stránky knihovny}
\url{http://knihovna.pedf.cuni.cz}
\subsection{Elektronický katalog}
\url{http://ckis.cuni.cz}
\subsection{Portál elektronických zdrojů UK}
\url{http://pez.cuni.cz}

\end{page}
\end{document}
